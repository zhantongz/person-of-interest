{\sloppy
\section{Package personOfInterest}{
\label{personOfInterest}\hskip -.05in
\hbox to \hsize{\textit{ Package Contents\hfil Page}}
\vskip .13in
\hbox{{\bf  Classes}}
\entityintro{Agent}{personOfInterest.Agent}{Agent of the organization}
\entityintro{CaughtInTheAct}{personOfInterest.CaughtInTheAct}{}
\entityintro{Game}{personOfInterest.Game}{Main game program}
\entityintro{Location}{personOfInterest.Location}{Location}
\entityintro{Mission}{personOfInterest.Mission}{Mission of the player}
\entityintro{Person}{personOfInterest.Person}{Person including POI, agents of the organization, and others}
\entityintro{Player}{personOfInterest.Player}{Player who play the game in the virtual world}
\entityintro{POI}{personOfInterest.POI}{Persons of Interest}
\entityintro{Problem}{personOfInterest.Problem}{Problem that will occur in a mission}
\entityintro{Saved}{personOfInterest.Saved}{Saved file for the game}
\vskip .1in
\vskip .1in
\section{\label{personOfInterest.Agent}\index{Agent}Class Agent}{
\vskip .1in 
Agent of the organization\\
ref: \url{http://docs.oracle.com/javase/tutorial/java/javaOO/index.html}\vskip .1in 
\subsection{Declaration}{
\small public class Agent
\\ {\bf  extends} personOfInterest.Person
\refdefined{personOfInterest.Person}}
\subsection{All known subclasses}{Player\small{\refdefined{personOfInterest.Player}}}
\subsection{Field summary}{
\begin{verse}
{\bf base} base location of an agent\\
{\bf calendar} calendar of an agent\\
{\bf currentDay} current day in the calendar\\
{\bf id} an agent's id number\\
{\bf rank} rank of an agent\\
{\bf RANKS} ranks' list\\ref: \url{http://java.about.com/od/understandingdatatypes/a/Using-Constants.htm}\\
{\bf serialVersionUID} serialization id\\
\end{verse}
}
\subsection{Constructor summary}{
\begin{verse}
{\bf Agent(String, int, Location)} Construct a new agent with specified name, rank and base location and a random id\\
{\bf Agent(String, int, Location, int)} Construct a new agent with specified name, rank and base location and a random id\\
\end{verse}
}
\subsection{Method summary}{
\begin{verse}
{\bf displayInfo()} Display an agent's info including rank, name, base and ID\\
{\bf passDay()} Pass current day\\
{\bf passDays(int)} Pass a certain number of days\\
{\bf resetCal()} Reset calendar of an agent\\
{\bf resetCal(int)} Reset calendar of an agent to a certain period\\
\end{verse}
}
\subsection{Fields}{
\begin{itemize}
\item{
\index{serialVersionUID}
\label{personOfInterest.Agent.serialVersionUID}private static final long {\bf  serialVersionUID}\begin{itemize}
\item{\vskip -.9ex 
serialization id}
\end{itemize}
}
\item{
\index{base}
\label{personOfInterest.Agent.base} Location {\bf  base}\begin{itemize}
\item{\vskip -.9ex 
base location of an agent}
\end{itemize}
}
\item{
\index{id}
\label{personOfInterest.Agent.id} int {\bf  id}\begin{itemize}
\item{\vskip -.9ex 
an agent's id number}
\end{itemize}
}
\item{
\index{rank}
\label{personOfInterest.Agent.rank} int {\bf  rank}\begin{itemize}
\item{\vskip -.9ex 
rank of an agent}
\end{itemize}
}
\item{
\index{RANKS}
\label{personOfInterest.Agent.RANKS}public static final java.lang.String {\bf  RANKS}\begin{itemize}
\item{\vskip -.9ex 
ranks' list\\ref: \url{http://java.about.com/od/understandingdatatypes/a/Using-Constants.htm}}
\end{itemize}
}
\item{
\index{calendar}
\label{personOfInterest.Agent.calendar} int {\bf  calendar}\begin{itemize}
\item{\vskip -.9ex 
calendar of an agent}
\end{itemize}
}
\item{
\index{currentDay}
\label{personOfInterest.Agent.currentDay} int {\bf  currentDay}\begin{itemize}
\item{\vskip -.9ex 
current day in the calendar}
\end{itemize}
}
\end{itemize}
}
\subsection{Constructors}{
\vskip -2em
\begin{itemize}
\item{ 
\index{Agent(String, int, Location)}
{\bf  Agent}\\
\texttt{public\ {\bf  Agent}(\texttt{java.lang.String} {\bf  inName},
\texttt{int} {\bf  inRank},
\texttt{Location} {\bf  inBase})
\label{personOfInterest.Agent(java.lang.String, int, personOfInterest.Location)}}%end signature
\begin{itemize}
\item{
{\bf  Description}

Construct a new agent with specified name, rank and base location and a random id
}
\item{
{\bf  Parameters}
  \begin{itemize}
   \item{
\texttt{inName} -- specified name for the new agent}
   \item{
\texttt{inRank} -- specified rank for the new agent}
   \item{
\texttt{inBase} -- specified base location for the new agent}
  \end{itemize}
}%end item
\end{itemize}
}%end item
\item{ 
\index{Agent(String, int, Location, int)}
{\bf  Agent}\\
\texttt{public\ {\bf  Agent}(\texttt{java.lang.String} {\bf  inName},
\texttt{int} {\bf  inRank},
\texttt{Location} {\bf  inBase},
\texttt{int} {\bf  inId})
\label{personOfInterest.Agent(java.lang.String, int, personOfInterest.Location, int)}}%end signature
\begin{itemize}
\item{
{\bf  Description}

Construct a new agent with specified name, rank and base location and a random id
}
\item{
{\bf  Parameters}
  \begin{itemize}
   \item{
\texttt{inName} -- specified name for the new agent}
   \item{
\texttt{inRank} -- specified rank for the new agent}
   \item{
\texttt{inBase} -- specified base location for the new agent}
   \item{
\texttt{inId} -- specified id number for the new agent}
  \end{itemize}
}%end item
\end{itemize}
}%end item
\end{itemize}
}
\subsection{Methods}{
\vskip -2em
\begin{itemize}
\item{ 
\index{displayInfo()}
{\bf  displayInfo}\\
\texttt{public void\ {\bf  displayInfo}()
\label{personOfInterest.Agent.displayInfo()}}%end signature
\begin{itemize}
\item{
{\bf  Description}

Display an agent's info including rank, name, base and ID
}
\end{itemize}
}%end item
\item{ 
\index{passDay()}
{\bf  passDay}\\
\texttt{public void\ {\bf  passDay}()
\label{personOfInterest.Agent.passDay()}}%end signature
\begin{itemize}
\item{
{\bf  Description}

Pass current day
}
\end{itemize}
}%end item
\item{ 
\index{passDays(int)}
{\bf  passDays}\\
\texttt{public void\ {\bf  passDays}(\texttt{int} {\bf  days})
\label{personOfInterest.Agent.passDays(int)}}%end signature
\begin{itemize}
\item{
{\bf  Description}

Pass a certain number of days
}
\item{
{\bf  Parameters}
  \begin{itemize}
   \item{
\texttt{days} -- the number of days passed}
  \end{itemize}
}%end item
\end{itemize}
}%end item
\item{ 
\index{resetCal()}
{\bf  resetCal}\\
\texttt{public void\ {\bf  resetCal}()
\label{personOfInterest.Agent.resetCal()}}%end signature
\begin{itemize}
\item{
{\bf  Description}

Reset calendar of an agent
}
\end{itemize}
}%end item
\item{ 
\index{resetCal(int)}
{\bf  resetCal}\\
\texttt{public void\ {\bf  resetCal}(\texttt{int} {\bf  days})
\label{personOfInterest.Agent.resetCal(int)}}%end signature
\begin{itemize}
\item{
{\bf  Description}

Reset calendar of an agent to a certain period
}
\end{itemize}
}%end item
\end{itemize}
}
\subsection{Members inherited from class Person }{
\texttt{personOfInterest.Person} {\small 
\refdefined{personOfInterest.Person}}
{\small 

\vskip -2em
\begin{itemize}
\item{\vskip -1.5ex 
\texttt{public void {\bf  displayInfo}()
}%end signature
}%end item
\item{\vskip -1.5ex 
\texttt{ {\bf  ifAlive}}%end signature
}%end item
\item{\vskip -1.5ex 
\texttt{public void {\bf  kill}()
}%end signature
}%end item
\item{\vskip -1.5ex 
\texttt{ {\bf  location}}%end signature
}%end item
\item{\vskip -1.5ex 
\texttt{ {\bf  name}}%end signature
}%end item
\item{\vskip -1.5ex 
\texttt{private static final {\bf  serialVersionUID}}%end signature
}%end item
\end{itemize}
}
}
\section{\label{personOfInterest.CaughtInTheAct}\index{CaughtInTheAct}Class CaughtInTheAct}{
\vskip .1in 
\subsection{Declaration}{
\small  class CaughtInTheAct
\\ {\bf  extends} personOfInterest.Mission
\refdefined{personOfInterest.Mission}}
\subsection{Constructor summary}{
\begin{verse}
{\bf CaughtInTheAct(POI, String, Location, int, int)} \\
\end{verse}
}
\subsection{Method summary}{
\begin{verse}
{\bf minorSituation(int)} \\
\end{verse}
}
\subsection{Constructors}{
\vskip -2em
\begin{itemize}
\item{ 
\index{CaughtInTheAct(POI, String, Location, int, int)}
{\bf  CaughtInTheAct}\\
\texttt{public\ {\bf  CaughtInTheAct}(\texttt{POI} {\bf  person},
\texttt{java.lang.String} {\bf  inname},
\texttt{Location} {\bf  inloc},
\texttt{int} {\bf  days},
\texttt{int} {\bf  inPoints}) throws java.io.FileNotFoundException, java.io.IOException, java.lang.ClassNotFoundException
\label{personOfInterest.CaughtInTheAct(personOfInterest.POI, java.lang.String, personOfInterest.Location, int, int)}}%end signature
}%end item
\end{itemize}
}
\subsection{Methods}{
\vskip -2em
\begin{itemize}
\item{ 
\index{minorSituation(int)}
{\bf  minorSituation}\\
\texttt{public void\ {\bf  minorSituation}(\texttt{int} {\bf  index}) throws java.lang.InterruptedException
\label{personOfInterest.CaughtInTheAct.minorSituation(int)}}%end signature
\begin{itemize}
\item{
{\bf  Description copied from Mission{\small \refdefined{personOfInterest.Mission}} }

Execute a customized minor situation
}
\item{
{\bf  Parameters}
  \begin{itemize}
   \item{
\texttt{index} -- the index of the minor situation}
  \end{itemize}
}%end item
\end{itemize}
}%end item
\end{itemize}
}
\subsection{Members inherited from class Mission }{
\texttt{personOfInterest.Mission} {\small 
\refdefined{personOfInterest.Mission}}
{\small 

\vskip -2em
\begin{itemize}
\item{\vskip -1.5ex 
\texttt{ {\bf  arriveMessage}}%end signature
}%end item
\item{\vskip -1.5ex 
\texttt{ {\bf  budget}}%end signature
}%end item
\item{\vskip -1.5ex 
\texttt{ {\bf  calendar}}%end signature
}%end item
\item{\vskip -1.5ex 
\texttt{public void {\bf  complete}()
}%end signature
}%end item
\item{\vskip -1.5ex 
\texttt{ {\bf  completed}}%end signature
}%end item
\item{\vskip -1.5ex 
\texttt{public boolean {\bf  completing}() throws java.lang.InterruptedException, java.lang.ClassNotFoundException, java.io.IOException
}%end signature
}%end item
\item{\vskip -1.5ex 
\texttt{ {\bf  dangers}}%end signature
}%end item
\item{\vskip -1.5ex 
\texttt{ {\bf  description}}%end signature
}%end item
\item{\vskip -1.5ex 
\texttt{public static Mission {\bf  generateFinalMission}(\texttt{int} {\bf  area}) throws java.io.FileNotFoundException, java.io.IOException, java.lang.ClassNotFoundException
}%end signature
}%end item
\item{\vskip -1.5ex 
\texttt{public static Mission {\bf  generateMission}(\texttt{int} {\bf  area},
\texttt{int} {\bf  rank}) throws java.io.FileNotFoundException, java.io.IOException, java.lang.ClassNotFoundException
}%end signature
}%end item
\item{\vskip -1.5ex 
\texttt{ {\bf  location}}%end signature
}%end item
\item{\vskip -1.5ex 
\texttt{ {\bf  minorNum}}%end signature
}%end item
\item{\vskip -1.5ex 
\texttt{public void {\bf  minorSituation}(\texttt{int} {\bf  index}) throws java.lang.InterruptedException
}%end signature
}%end item
\item{\vskip -1.5ex 
\texttt{ {\bf  name}}%end signature
}%end item
\item{\vskip -1.5ex 
\texttt{ {\bf  player}}%end signature
}%end item
\item{\vskip -1.5ex 
\texttt{ {\bf  poi}}%end signature
}%end item
\item{\vskip -1.5ex 
\texttt{ {\bf  points}}%end signature
}%end item
\item{\vskip -1.5ex 
\texttt{ {\bf  problems}}%end signature
}%end item
\item{\vskip -1.5ex 
\texttt{ {\bf  rank}}%end signature
}%end item
\item{\vskip -1.5ex 
\texttt{public static String {\bf  replaceName}(\texttt{java.lang.String} {\bf  s},
\texttt{Mission} {\bf  mission})
}%end signature
}%end item
\end{itemize}
}
}
\section{\label{personOfInterest.Game}\index{Game}Class Game}{
\vskip .1in 
Main game program\vskip .1in 
\subsection{Declaration}{
\small public class Game
\\ {\bf  extends} java.lang.Object
\refdefined{java.lang.Object}}
\subsection{Field summary}{
\begin{verse}
{\bf gameCal} game calendar\\
{\bf input} scanner that reads from keyboard\\
{\bf locArea} \\
{\bf newMission} if the player wants to start a new mission\\
{\bf poiArea} \\
{\bf savedName} saved name for saving game\\
{\bf savedPath} saved path for saving game\\
\end{verse}
}
\subsection{Constructor summary}{
\begin{verse}
{\bf Game()} \\
\end{verse}
}
\subsection{Method summary}{
\begin{verse}
{\bf addLinebreaks(String)} Break lines in a long string with maximum length of 80\\
{\bf addLinebreaks(String, int)} Break lines in a long string to make display nicer\\ref: \url{http://stackoverflow.com/questions/7528045/large-string-split-into-lines-with-maximum-length-in-java}\\
{\bf displayDate()} Display the date today (as in the game)\\
{\bf help()} Display help information\\
{\bf ifGoodCalendar(int\lbrack \rbrack )} Determine if a player complete his/her mission in a certain time\\
{\bf input(String)} Get the user's input with space as the delimiter\\
{\bf input(String, int, int)} Check if the user's input is valid and return a valid data\\
{\bf inputLn(String)} Get the user's input with line break as the delimiter\\
{\bf isGameOver(Player)} Determine if the game is over\\
{\bf loadFile(String)} Load a serialized file\\
{\bf locs()} Update location information from the csv file\\
{\bf main(String\lbrack \rbrack )} Main program\\
{\bf msts()} Update location information from the csv file\\
{\bf passDay()} Add 1 day to the calendar\\
{\bf passDays(int)} Add a certain number of days to the game calendar\\
{\bf pois()} Update location information from the csv file\\
{\bf prbs()} Update problems\\
{\bf processUserInput(Problem, Player)} Process the user's input that intends for a problem\\
{\bf processUserInput(String, Player)} Process the user's input\\
{\bf randomID()} Generate a 9-digits id number for agent id, mission id, etc.\\
{\bf randomNum(int, int)} Generate a random number in a certain range\\
{\bf saveGame(Player)} Save the game\\
{\bf scciSeal()} Print SCCI logo\\
{\bf showStartPage()} Show the start page with ASCII art\\
{\bf sleep(int)} Pause the program for a period of time while outputing three dots\\ref: \url{http://docs.oracle.com/javase/tutorial/essential/concurrency/sleep.html}\\
{\bf takeChance(double)} \\
{\bf takeChance(int)} \\
{\bf toBoolean(int)} Cast int to boolean\\
{\bf transport(Agent, int, Location, Location)} Transport an agent from his/her current location to a location requested through a certain way\\
{\bf typeString(String)} Show a string in typewriter style\\
\end{verse}
}
\subsection{Fields}{
\begin{itemize}
\item{
\index{input}
\label{personOfInterest.Game.input}static java.util.Scanner {\bf  input}\begin{itemize}
\item{\vskip -.9ex 
scanner that reads from keyboard}
\end{itemize}
}
\item{
\index{savedPath}
\label{personOfInterest.Game.savedPath}static java.lang.String {\bf  savedPath}\begin{itemize}
\item{\vskip -.9ex 
saved path for saving game}
\end{itemize}
}
\item{
\index{savedName}
\label{personOfInterest.Game.savedName}static java.lang.String {\bf  savedName}\begin{itemize}
\item{\vskip -.9ex 
saved name for saving game}
\end{itemize}
}
\item{
\index{gameCal}
\label{personOfInterest.Game.gameCal}static java.util.Calendar {\bf  gameCal}\begin{itemize}
\item{\vskip -.9ex 
game calendar}
\end{itemize}
}
\item{
\index{locArea}
\label{personOfInterest.Game.locArea}static int {\bf  locArea}}
\item{
\index{poiArea}
\label{personOfInterest.Game.poiArea}static int {\bf  poiArea}}
\item{
\index{newMission}
\label{personOfInterest.Game.newMission}static boolean {\bf  newMission}\begin{itemize}
\item{\vskip -.9ex 
if the player wants to start a new mission}
\end{itemize}
}
\end{itemize}
}
\subsection{Constructors}{
\vskip -2em
\begin{itemize}
\item{ 
\index{Game()}
{\bf  Game}\\
\texttt{public\ {\bf  Game}()
\label{personOfInterest.Game()}}%end signature
}%end item
\end{itemize}
}
\subsection{Methods}{
\vskip -2em
\begin{itemize}
\item{ 
\index{addLinebreaks(String)}
{\bf  addLinebreaks}\\
\texttt{public static java.lang.String\ {\bf  addLinebreaks}(\texttt{java.lang.String} {\bf  input})
\label{personOfInterest.Game.addLinebreaks(java.lang.String)}}%end signature
\begin{itemize}
\item{
{\bf  Description}

Break lines in a long string with maximum length of 80
}
\item{
{\bf  Parameters}
  \begin{itemize}
   \item{
\texttt{input} -- the string need to be broken}
  \end{itemize}
}%end item
\item{{\bf  Returns} -- 
the string with proper breaks 
}%end item
\end{itemize}
}%end item
\item{ 
\index{addLinebreaks(String, int)}
{\bf  addLinebreaks}\\
\texttt{public static java.lang.String\ {\bf  addLinebreaks}(\texttt{java.lang.String} {\bf  input},
\texttt{int} {\bf  maxLineLength})
\label{personOfInterest.Game.addLinebreaks(java.lang.String, int)}}%end signature
\begin{itemize}
\item{
{\bf  Description}

Break lines in a long string to make display nicer\\ref: \url{http://stackoverflow.com/questions/7528045/large-string-split-into- lines-with-maximum-length-in-java}
}
\item{
{\bf  Parameters}
  \begin{itemize}
   \item{
\texttt{input} -- the string need to be broken}
   \item{
\texttt{maxLineLength} -- the maxium line length to insert break line}
  \end{itemize}
}%end item
\item{{\bf  Returns} -- 
the string with proper breaks 
}%end item
\end{itemize}
}%end item
\item{ 
\index{displayDate()}
{\bf  displayDate}\\
\texttt{public static java.lang.String\ {\bf  displayDate}()
\label{personOfInterest.Game.displayDate()}}%end signature
\begin{itemize}
\item{
{\bf  Description}

Display the date today (as in the game)
}
\item{{\bf  Returns} -- 
the current date in the game 
}%end item
\end{itemize}
}%end item
\item{ 
\index{help()}
{\bf  help}\\
\texttt{public static void\ {\bf  help}() throws java.io.FileNotFoundException
\label{personOfInterest.Game.help()}}%end signature
\begin{itemize}
\item{
{\bf  Description}

Display help information
}
\end{itemize}
}%end item
\item{ 
\index{ifGoodCalendar(int\lbrack \rbrack )}
{\bf  ifGoodCalendar}\\
\texttt{public static boolean\ {\bf  ifGoodCalendar}(\texttt{int\lbrack \rbrack } {\bf  calendar})
\label{personOfInterest.Game.ifGoodCalendar(int[])}}%end signature
\begin{itemize}
\item{
{\bf  Description}

Determine if a player complete his/her mission in a certain time
}
\item{
{\bf  Parameters}
  \begin{itemize}
   \item{
\texttt{calendar} -- the calendar of the player/mission}
  \end{itemize}
}%end item
\item{{\bf  Returns} -- 
true if the player doesn't fail to comply the calendar; false otherwise 
}%end item
\end{itemize}
}%end item
\item{ 
\index{input(String)}
{\bf  input}\\
\texttt{public static java.lang.String\ {\bf  input}(\texttt{java.lang.String} {\bf  prompt}) throws java.io.IOException
\label{personOfInterest.Game.input(java.lang.String)}}%end signature
\begin{itemize}
\item{
{\bf  Description}

Get the user's input with space as the delimiter
}
\item{
{\bf  Parameters}
  \begin{itemize}
   \item{
\texttt{prompt} -- prompt for the user}
  \end{itemize}
}%end item
\item{{\bf  Returns} -- 
the string user entered 
}%end item
\end{itemize}
}%end item
\item{ 
\index{input(String, int, int)}
{\bf  input}\\
\texttt{public static int\ {\bf  input}(\texttt{java.lang.String} {\bf  prompt},
\texttt{int} {\bf  min},
\texttt{int} {\bf  max})
\label{personOfInterest.Game.input(java.lang.String, int, int)}}%end signature
\begin{itemize}
\item{
{\bf  Description}

Check if the user's input is valid and return a valid data
}
\item{
{\bf  Parameters}
  \begin{itemize}
   \item{
\texttt{prompt} -- prompt for the user}
   \item{
\texttt{min} -- the minimum value for a valid data}
   \item{
\texttt{max} -- the maximum value for a valid data}
  \end{itemize}
}%end item
\item{{\bf  Returns} -- 
a valid integer data between min and max 
}%end item
\end{itemize}
}%end item
\item{ 
\index{inputLn(String)}
{\bf  inputLn}\\
\texttt{public static java.lang.String\ {\bf  inputLn}(\texttt{java.lang.String} {\bf  prompt})
\label{personOfInterest.Game.inputLn(java.lang.String)}}%end signature
\begin{itemize}
\item{
{\bf  Description}

Get the user's input with line break as the delimiter
}
\item{
{\bf  Parameters}
  \begin{itemize}
   \item{
\texttt{prompt} -- prompt for the user}
  \end{itemize}
}%end item
\item{{\bf  Returns} -- 
the string user entered 
}%end item
\end{itemize}
}%end item
\item{ 
\index{isGameOver(Player)}
{\bf  isGameOver}\\
\texttt{public static boolean\ {\bf  isGameOver}(\texttt{Player} {\bf  player})
\label{personOfInterest.Game.isGameOver(personOfInterest.Player)}}%end signature
\begin{itemize}
\item{
{\bf  Description}

Determine if the game is over
}
\item{{\bf  Returns} -- 
true if the player is alive and not completed the game; false otherwise 
}%end item
\end{itemize}
}%end item
\item{ 
\index{loadFile(String)}
{\bf  loadFile}\\
\texttt{public static java.lang.Object\ {\bf  loadFile}(\texttt{java.lang.String} {\bf  prompt}) throws java.io.IOException
\label{personOfInterest.Game.loadFile(java.lang.String)}}%end signature
\begin{itemize}
\item{
{\bf  Description}

Load a serialized file
}
\item{
{\bf  Parameters}
  \begin{itemize}
   \item{
\texttt{prompt} -- prompt for the user}
  \end{itemize}
}%end item
\item{{\bf  Returns} -- 
the object included in the file 
}%end item
\end{itemize}
}%end item
\item{ 
\index{locs()}
{\bf  locs}\\
\texttt{public static void\ {\bf  locs}() throws java.lang.NumberFormatException, java.io.IOException, java.lang.InterruptedException
\label{personOfInterest.Game.locs()}}%end signature
\begin{itemize}
\item{
{\bf  Description}

Update location information from the csv file
}
\end{itemize}
}%end item
\item{ 
\index{main(String\lbrack \rbrack )}
{\bf  main}\\
\texttt{public static void\ {\bf  main}(\texttt{java.lang.String\lbrack \rbrack } {\bf  args}) throws java.io.IOException, java.lang.ClassNotFoundException, java.lang.InterruptedException
\label{personOfInterest.Game.main(java.lang.String[])}}%end signature
\begin{itemize}
\item{
{\bf  Description}

Main program
}
\end{itemize}
}%end item
\item{ 
\index{msts()}
{\bf  msts}\\
\texttt{public static void\ {\bf  msts}() throws java.lang.NumberFormatException, java.io.IOException, java.lang.InterruptedException
\label{personOfInterest.Game.msts()}}%end signature
\begin{itemize}
\item{
{\bf  Description}

Update location information from the csv file
}
\end{itemize}
}%end item
\item{ 
\index{passDay()}
{\bf  passDay}\\
\texttt{public static void\ {\bf  passDay}()
\label{personOfInterest.Game.passDay()}}%end signature
\begin{itemize}
\item{
{\bf  Description}

Add 1 day to the calendar
}
\end{itemize}
}%end item
\item{ 
\index{passDays(int)}
{\bf  passDays}\\
\texttt{public static void\ {\bf  passDays}(\texttt{int} {\bf  days})
\label{personOfInterest.Game.passDays(int)}}%end signature
\begin{itemize}
\item{
{\bf  Description}

Add a certain number of days to the game calendar
}
\item{
{\bf  Parameters}
  \begin{itemize}
   \item{
\texttt{days} -- the certain number of days}
  \end{itemize}
}%end item
\end{itemize}
}%end item
\item{ 
\index{pois()}
{\bf  pois}\\
\texttt{public static void\ {\bf  pois}() throws java.lang.NumberFormatException, java.io.IOException, java.lang.InterruptedException
\label{personOfInterest.Game.pois()}}%end signature
\begin{itemize}
\item{
{\bf  Description}

Update location information from the csv file
}
\end{itemize}
}%end item
\item{ 
\index{prbs()}
{\bf  prbs}\\
\texttt{public static void\ {\bf  prbs}() throws java.io.IOException
\label{personOfInterest.Game.prbs()}}%end signature
\begin{itemize}
\item{
{\bf  Description}

Update problems
}
\end{itemize}
}%end item
\item{ 
\index{processUserInput(Problem, Player)}
{\bf  processUserInput}\\
\texttt{public static boolean\ {\bf  processUserInput}(\texttt{Problem} {\bf  problem},
\texttt{Player} {\bf  player}) throws java.io.IOException, java.lang.ClassNotFoundException, java.lang.InterruptedException
\label{personOfInterest.Game.processUserInput(personOfInterest.Problem, personOfInterest.Player)}}%end signature
\begin{itemize}
\item{
{\bf  Description}

Process the user's input that intends for a problem
}
\item{
{\bf  Parameters}
  \begin{itemize}
   \item{
\texttt{problem} -- the problem}
   \item{
\texttt{player} -- the player}
  \end{itemize}
}%end item
\item{{\bf  Returns} -- 
true if the user answers the question correctly; false if answers incorrectly or doesn't answer 
}%end item
\end{itemize}
}%end item
\item{ 
\index{processUserInput(String, Player)}
{\bf  processUserInput}\\
\texttt{public static void\ {\bf  processUserInput}(\texttt{java.lang.String} {\bf  uInput},
\texttt{Player} {\bf  player}) throws java.io.FileNotFoundException, java.io.IOException, java.lang.ClassNotFoundException, java.lang.InterruptedException
\label{personOfInterest.Game.processUserInput(java.lang.String, personOfInterest.Player)}}%end signature
\begin{itemize}
\item{
{\bf  Description}

Process the user's input
}
\item{
{\bf  Parameters}
  \begin{itemize}
   \item{
\texttt{uinput} -- the user's input}
   \item{
\texttt{player} -- the player}
  \end{itemize}
}%end item
\end{itemize}
}%end item
\item{ 
\index{randomID()}
{\bf  randomID}\\
\texttt{public static int\ {\bf  randomID}()
\label{personOfInterest.Game.randomID()}}%end signature
\begin{itemize}
\item{
{\bf  Description}

Generate a 9-digits id number for agent id, mission id, etc.
}
\item{{\bf  Returns} -- 
the generated 9-digits number 
}%end item
\end{itemize}
}%end item
\item{ 
\index{randomNum(int, int)}
{\bf  randomNum}\\
\texttt{public static int\ {\bf  randomNum}(\texttt{int} {\bf  min},
\texttt{int} {\bf  max})
\label{personOfInterest.Game.randomNum(int, int)}}%end signature
\begin{itemize}
\item{
{\bf  Description}

Generate a random number in a certain range
}
\item{
{\bf  Parameters}
  \begin{itemize}
   \item{
\texttt{min} -- the minimum for the number, inclusive}
   \item{
\texttt{max} -- the maximum for the number, inclusive}
  \end{itemize}
}%end item
\item{{\bf  Returns} -- 
the generated number 
}%end item
\end{itemize}
}%end item
\item{ 
\index{saveGame(Player)}
{\bf  saveGame}\\
\texttt{public static boolean\ {\bf  saveGame}(\texttt{Player} {\bf  player}) throws java.io.FileNotFoundException, java.io.IOException
\label{personOfInterest.Game.saveGame(personOfInterest.Player)}}%end signature
\begin{itemize}
\item{
{\bf  Description}

Save the game
}
\item{
{\bf  Parameters}
  \begin{itemize}
   \item{
\texttt{player} -- the game's player}
  \end{itemize}
}%end item
\item{{\bf  Returns} -- 
true if the saving is a success; false otherwise 
}%end item
\end{itemize}
}%end item
\item{ 
\index{scciSeal()}
{\bf  scciSeal}\\
\texttt{public static void\ {\bf  scciSeal}()
\label{personOfInterest.Game.scciSeal()}}%end signature
\begin{itemize}
\item{
{\bf  Description}

Print SCCI logo
}
\end{itemize}
}%end item
\item{ 
\index{showStartPage()}
{\bf  showStartPage}\\
\texttt{public static void\ {\bf  showStartPage}()
\label{personOfInterest.Game.showStartPage()}}%end signature
\begin{itemize}
\item{
{\bf  Description}

Show the start page with ASCII art
}
\end{itemize}
}%end item
\item{ 
\index{sleep(int)}
{\bf  sleep}\\
\texttt{public static void\ {\bf  sleep}(\texttt{int} {\bf  timeInMs}) throws java.lang.InterruptedException
\label{personOfInterest.Game.sleep(int)}}%end signature
\begin{itemize}
\item{
{\bf  Description}

Pause the program for a period of time while outputing three dots\\ref: \url{http://docs.oracle.com/javase/tutorial/essential/concurrency/sleep.html}
}
\item{
{\bf  Parameters}
  \begin{itemize}
   \item{
\texttt{timeInMs} -- the period of time to be paused for}
  \end{itemize}
}%end item
\end{itemize}
}%end item
\item{ 
\index{takeChance(double)}
{\bf  takeChance}\\
\texttt{public static boolean\ {\bf  takeChance}(\texttt{double} {\bf  chance})
\label{personOfInterest.Game.takeChance(double)}}%end signature
}%end item
\item{ 
\index{takeChance(int)}
{\bf  takeChance}\\
\texttt{public static boolean\ {\bf  takeChance}(\texttt{int} {\bf  divisor})
\label{personOfInterest.Game.takeChance(int)}}%end signature
}%end item
\item{ 
\index{toBoolean(int)}
{\bf  toBoolean}\\
\texttt{public static boolean\ {\bf  toBoolean}(\texttt{int} {\bf  num})
\label{personOfInterest.Game.toBoolean(int)}}%end signature
\begin{itemize}
\item{
{\bf  Description}

Cast int to boolean
}
\item{
{\bf  Parameters}
  \begin{itemize}
   \item{
\texttt{num} -- the int ready to be casted}
  \end{itemize}
}%end item
\item{{\bf  Returns} -- 
false if the int is 0; true otherwise 
}%end item
\end{itemize}
}%end item
\item{ 
\index{transport(Agent, int, Location, Location)}
{\bf  transport}\\
\texttt{public static void\ {\bf  transport}(\texttt{Agent} {\bf  agent},
\texttt{int} {\bf  method},
\texttt{Location} {\bf  from},
\texttt{Location} {\bf  to})
\label{personOfInterest.Game.transport(personOfInterest.Agent, int, personOfInterest.Location, personOfInterest.Location)}}%end signature
\begin{itemize}
\item{
{\bf  Description}

Transport an agent from his/her current location to a location requested through a certain way
}
\item{
{\bf  Parameters}
  \begin{itemize}
   \item{
\texttt{agent} -- the agent that need to be transported}
   \item{
\texttt{method} -- the way of transportation; 1 for train, 2 for plane, 3 for normal vehicle and 4 for high-speed rail}
   \item{
\texttt{from} -- the current location}
   \item{
\texttt{to} -- the location requested}
  \end{itemize}
}%end item
\end{itemize}
}%end item
\item{ 
\index{typeString(String)}
{\bf  typeString}\\
\texttt{public static void\ {\bf  typeString}(\texttt{java.lang.String} {\bf  message}) throws java.lang.InterruptedException
\label{personOfInterest.Game.typeString(java.lang.String)}}%end signature
\begin{itemize}
\item{
{\bf  Description}

Show a string in typewriter style
}
\item{
{\bf  Parameters}
  \begin{itemize}
   \item{
\texttt{message} -- the string need to be displayed}
  \end{itemize}
}%end item
\end{itemize}
}%end item
\end{itemize}
}
}
\section{\label{personOfInterest.Location}\index{Location}Class Location}{
\vskip .1in 
Location\vskip .1in 
\subsection{Declaration}{
\small public class Location
\\ {\bf  extends} java.lang.Object
\refdefined{java.lang.Object}\\ {\bf  implements} 
java.io.Serializable}
\subsection{Field summary}{
\begin{verse}
{\bf area} a location's area as the index in AREAS\\
{\bf AREAS} areas that a location can be included in\\
{\bf city} city that a location situated\\
{\bf country} country that a location situated\\
{\bf description} description of a location\\
{\bf name} a location's name\\
{\bf serialVersionUID} serialization id\\
{\bf sublocations} locations that are in a location\\
{\bf subNum} the number of sublocations\\
\end{verse}
}
\subsection{Constructor summary}{
\begin{verse}
{\bf Location(String, String, String, int, String)} Construct a location with name.\\
\end{verse}
}
\subsection{Method summary}{
\begin{verse}
{\bf addSublocation(Location)} Add a sublocation\\
{\bf addSublocations(Location\lbrack \rbrack )} Add a series of sublocation\\
\end{verse}
}
\subsection{Fields}{
\begin{itemize}
\item{
\index{serialVersionUID}
\label{personOfInterest.Location.serialVersionUID}private static final long {\bf  serialVersionUID}\begin{itemize}
\item{\vskip -.9ex 
serialization id}
\end{itemize}
}
\item{
\index{name}
\label{personOfInterest.Location.name} java.lang.String {\bf  name}\begin{itemize}
\item{\vskip -.9ex 
a location's name}
\end{itemize}
}
\item{
\index{area}
\label{personOfInterest.Location.area} int {\bf  area}\begin{itemize}
\item{\vskip -.9ex 
a location's area as the index in AREAS}
\end{itemize}
}
\item{
\index{city}
\label{personOfInterest.Location.city} java.lang.String {\bf  city}\begin{itemize}
\item{\vskip -.9ex 
city that a location situated}
\end{itemize}
}
\item{
\index{country}
\label{personOfInterest.Location.country} java.lang.String {\bf  country}\begin{itemize}
\item{\vskip -.9ex 
country that a location situated}
\end{itemize}
}
\item{
\index{sublocations}
\label{personOfInterest.Location.sublocations} Location {\bf  sublocations}\begin{itemize}
\item{\vskip -.9ex 
locations that are in a location}
\end{itemize}
}
\item{
\index{subNum}
\label{personOfInterest.Location.subNum} int {\bf  subNum}\begin{itemize}
\item{\vskip -.9ex 
the number of sublocations}
\end{itemize}
}
\item{
\index{description}
\label{personOfInterest.Location.description} java.lang.String {\bf  description}\begin{itemize}
\item{\vskip -.9ex 
description of a location}
\end{itemize}
}
\item{
\index{AREAS}
\label{personOfInterest.Location.AREAS}static final java.lang.String {\bf  AREAS}\begin{itemize}
\item{\vskip -.9ex 
areas that a location can be included in}
\end{itemize}
}
\end{itemize}
}
\subsection{Constructors}{
\vskip -2em
\begin{itemize}
\item{ 
\index{Location(String, String, String, int, String)}
{\bf  Location}\\
\texttt{public\ {\bf  Location}(\texttt{java.lang.String} {\bf  inName},
\texttt{java.lang.String} {\bf  inCity},
\texttt{java.lang.String} {\bf  inCountry},
\texttt{int} {\bf  inArea},
\texttt{java.lang.String} {\bf  inDescription})
\label{personOfInterest.Location(java.lang.String, java.lang.String, java.lang.String, int, java.lang.String)}}%end signature
\begin{itemize}
\item{
{\bf  Description}

Construct a location with name.
}
\item{
{\bf  Parameters}
  \begin{itemize}
   \item{
\texttt{inName} -- the location's name}
  \end{itemize}
}%end item
\end{itemize}
}%end item
\end{itemize}
}
\subsection{Methods}{
\vskip -2em
\begin{itemize}
\item{ 
\index{addSublocation(Location)}
{\bf  addSublocation}\\
\texttt{public void\ {\bf  addSublocation}(\texttt{Location} {\bf  subloc})
\label{personOfInterest.Location.addSublocation(personOfInterest.Location)}}%end signature
\begin{itemize}
\item{
{\bf  Description}

Add a sublocation
}
\item{
{\bf  Parameters}
  \begin{itemize}
   \item{
\texttt{subloc} -- the sublocation that needs to be added}
  \end{itemize}
}%end item
\end{itemize}
}%end item
\item{ 
\index{addSublocations(Location\lbrack \rbrack )}
{\bf  addSublocations}\\
\texttt{public void\ {\bf  addSublocations}(\texttt{Location\lbrack \rbrack } {\bf  sublocs})
\label{personOfInterest.Location.addSublocations(personOfInterest.Location[])}}%end signature
\begin{itemize}
\item{
{\bf  Description}

Add a series of sublocation
}
\item{
{\bf  Parameters}
  \begin{itemize}
   \item{
\texttt{sublocs} -- the sublocations that needs to be added}
  \end{itemize}
}%end item
\end{itemize}
}%end item
\end{itemize}
}
}
\section{\label{personOfInterest.Mission}\index{Mission}Class Mission}{
\vskip .1in 
Mission of the player\vskip .1in 
\subsection{Declaration}{
\small public class Mission
\\ {\bf  extends} java.lang.Object
\refdefined{java.lang.Object}}
\subsection{All known subclasses}{CaughtInTheAct\small{\refdefined{personOfInterest.CaughtInTheAct}}}
\subsection{Field summary}{
\begin{verse}
{\bf arriveMessage} message displayed after arrival\\
{\bf budget} money that can be used to complete a mission\\
{\bf calendar} calendar of a mission; time permitted\\
{\bf completed} if a mission is completed\\
{\bf dangers} potential dangers of a mission\\
{\bf description} description of a mission\\
{\bf location} location that a mission takes place\\
{\bf minorNum} indication of where minor situations happens\\
{\bf name} name of a mission\\
{\bf player} player who is completing a mission\\
{\bf poi} person of interest in a mission\\
{\bf points} points that awarded after completion\\
{\bf problems} situations in a mission\\
{\bf rank} rank that have abilities to complete a mission\\
\end{verse}
}
\subsection{Constructor summary}{
\begin{verse}
{\bf Mission(POI, String, Location, int, int)} Construct a mission\\
\end{verse}
}
\subsection{Method summary}{
\begin{verse}
{\bf complete()} Complete a mission\\
{\bf completing()} Process a mission\\
{\bf generateFinalMission(int)} Generate a final Mission\\
{\bf generateMission(int, int)} Generate a random mission\\
{\bf minorSituation(int)} Execute a customized minor situation\\
{\bf replaceName(String, Mission)} Replace certain tags in a string with certain variables\\
\end{verse}
}
\subsection{Fields}{
\begin{itemize}
\item{
\index{location}
\label{personOfInterest.Mission.location} Location {\bf  location}\begin{itemize}
\item{\vskip -.9ex 
location that a mission takes place}
\end{itemize}
}
\item{
\index{name}
\label{personOfInterest.Mission.name} java.lang.String {\bf  name}\begin{itemize}
\item{\vskip -.9ex 
name of a mission}
\end{itemize}
}
\item{
\index{poi}
\label{personOfInterest.Mission.poi} POI {\bf  poi}\begin{itemize}
\item{\vskip -.9ex 
person of interest in a mission}
\end{itemize}
}
\item{
\index{completed}
\label{personOfInterest.Mission.completed} boolean {\bf  completed}\begin{itemize}
\item{\vskip -.9ex 
if a mission is completed}
\end{itemize}
}
\item{
\index{rank}
\label{personOfInterest.Mission.rank} int {\bf  rank}\begin{itemize}
\item{\vskip -.9ex 
rank that have abilities to complete a mission}
\end{itemize}
}
\item{
\index{budget}
\label{personOfInterest.Mission.budget} int {\bf  budget}\begin{itemize}
\item{\vskip -.9ex 
money that can be used to complete a mission}
\end{itemize}
}
\item{
\index{calendar}
\label{personOfInterest.Mission.calendar} int {\bf  calendar}\begin{itemize}
\item{\vskip -.9ex 
calendar of a mission; time permitted}
\end{itemize}
}
\item{
\index{points}
\label{personOfInterest.Mission.points} int {\bf  points}\begin{itemize}
\item{\vskip -.9ex 
points that awarded after completion}
\end{itemize}
}
\item{
\index{player}
\label{personOfInterest.Mission.player} Player {\bf  player}\begin{itemize}
\item{\vskip -.9ex 
player who is completing a mission}
\end{itemize}
}
\item{
\index{dangers}
\label{personOfInterest.Mission.dangers} java.lang.String {\bf  dangers}\begin{itemize}
\item{\vskip -.9ex 
potential dangers of a mission}
\end{itemize}
}
\item{
\index{description}
\label{personOfInterest.Mission.description} java.lang.String {\bf  description}\begin{itemize}
\item{\vskip -.9ex 
description of a mission}
\end{itemize}
}
\item{
\index{problems}
\label{personOfInterest.Mission.problems} Problem {\bf  problems}\begin{itemize}
\item{\vskip -.9ex 
situations in a mission}
\end{itemize}
}
\item{
\index{arriveMessage}
\label{personOfInterest.Mission.arriveMessage} java.lang.String {\bf  arriveMessage}\begin{itemize}
\item{\vskip -.9ex 
message displayed after arrival}
\end{itemize}
}
\item{
\index{minorNum}
\label{personOfInterest.Mission.minorNum} int {\bf  minorNum}\begin{itemize}
\item{\vskip -.9ex 
indication of where minor situations happens}
\end{itemize}
}
\end{itemize}
}
\subsection{Constructors}{
\vskip -2em
\begin{itemize}
\item{ 
\index{Mission(POI, String, Location, int, int)}
{\bf  Mission}\\
\texttt{public\ {\bf  Mission}(\texttt{POI} {\bf  person},
\texttt{java.lang.String} {\bf  inname},
\texttt{Location} {\bf  inloc},
\texttt{int} {\bf  days},
\texttt{int} {\bf  inPoints})
\label{personOfInterest.Mission(personOfInterest.POI, java.lang.String, personOfInterest.Location, int, int)}}%end signature
\begin{itemize}
\item{
{\bf  Description}

Construct a mission
}
\item{
{\bf  Parameters}
  \begin{itemize}
   \item{
\texttt{person} -- the POI in a mission}
   \item{
\texttt{inname} -- the name of a mission}
   \item{
\texttt{inloc} -- the location of a mission}
   \item{
\texttt{days} -- the time limit in days for the mission}
   \item{
\texttt{inPoints} -- the points required to complete the mission}
  \end{itemize}
}%end item
\end{itemize}
}%end item
\end{itemize}
}
\subsection{Methods}{
\vskip -2em
\begin{itemize}
\item{ 
\index{complete()}
{\bf  complete}\\
\texttt{public void\ {\bf  complete}()
\label{personOfInterest.Mission.complete()}}%end signature
\begin{itemize}
\item{
{\bf  Description}

Complete a mission
}
\item{
{\bf  Parameters}
  \begin{itemize}
   \item{
\texttt{player} -- the player who completed the mission}
  \end{itemize}
}%end item
\end{itemize}
}%end item
\item{ 
\index{completing()}
{\bf  completing}\\
\texttt{public boolean\ {\bf  completing}() throws java.lang.InterruptedException, java.lang.ClassNotFoundException, java.io.IOException
\label{personOfInterest.Mission.completing()}}%end signature
\begin{itemize}
\item{
{\bf  Description}

Process a mission
}
\item{{\bf  Returns} -- 
true if complete the mission; false otherwise 
}%end item
\end{itemize}
}%end item
\item{ 
\index{generateFinalMission(int)}
{\bf  generateFinalMission}\\
\texttt{public static Mission\ {\bf  generateFinalMission}(\texttt{int} {\bf  area}) throws java.io.FileNotFoundException, java.io.IOException, java.lang.ClassNotFoundException
\label{personOfInterest.Mission.generateFinalMission(int)}}%end signature
\begin{itemize}
\item{
{\bf  Description}

Generate a final Mission
}
\item{
{\bf  Parameters}
  \begin{itemize}
   \item{
\texttt{area} -- the area of the mission}
  \end{itemize}
}%end item
\item{{\bf  Returns} -- 
the generated final mission 
}%end item
\end{itemize}
}%end item
\item{ 
\index{generateMission(int, int)}
{\bf  generateMission}\\
\texttt{public static Mission\ {\bf  generateMission}(\texttt{int} {\bf  area},
\texttt{int} {\bf  rank}) throws java.io.FileNotFoundException, java.io.IOException, java.lang.ClassNotFoundException
\label{personOfInterest.Mission.generateMission(int, int)}}%end signature
\begin{itemize}
\item{
{\bf  Description}

Generate a random mission
}
\item{
{\bf  Parameters}
  \begin{itemize}
   \item{
\texttt{area} -- the area of the mission}
   \item{
\texttt{rank} -- the rank that the mission is designed for}
  \end{itemize}
}%end item
\item{{\bf  Returns} -- 
generated mission object 
}%end item
\end{itemize}
}%end item
\item{ 
\index{minorSituation(int)}
{\bf  minorSituation}\\
\texttt{public void\ {\bf  minorSituation}(\texttt{int} {\bf  index}) throws java.lang.InterruptedException
\label{personOfInterest.Mission.minorSituation(int)}}%end signature
\begin{itemize}
\item{
{\bf  Description}

Execute a customized minor situation
}
\item{
{\bf  Parameters}
  \begin{itemize}
   \item{
\texttt{index} -- the index of the minor situation}
  \end{itemize}
}%end item
\end{itemize}
}%end item
\item{ 
\index{replaceName(String, Mission)}
{\bf  replaceName}\\
\texttt{public static java.lang.String\ {\bf  replaceName}(\texttt{java.lang.String} {\bf  s},
\texttt{Mission} {\bf  mission})
\label{personOfInterest.Mission.replaceName(java.lang.String, personOfInterest.Mission)}}%end signature
\begin{itemize}
\item{
{\bf  Description}

Replace certain tags in a string with certain variables
}
\item{
{\bf  Parameters}
  \begin{itemize}
   \item{
\texttt{s} -- the string need to be replaced}
   \item{
\texttt{mission} -- the mission that indicate variables}
  \end{itemize}
}%end item
\item{{\bf  Returns} -- 
the string that the certain tags are replaced 
}%end item
\end{itemize}
}%end item
\end{itemize}
}
}
\section{\label{personOfInterest.Person}\index{Person}Class Person}{
\vskip .1in 
Person including POI, agents of the organization, and others\vskip .1in 
\subsection{Declaration}{
\small public class Person
\\ {\bf  extends} java.lang.Object
\refdefined{java.lang.Object}\\ {\bf  implements} 
java.io.Serializable}
\subsection{All known subclasses}{Player\small{\refdefined{personOfInterest.Player}}, Agent\small{\refdefined{personOfInterest.Agent}}, POI\small{\refdefined{personOfInterest.POI}}}
\subsection{Field summary}{
\begin{verse}
{\bf ifAlive} person's living status\\
{\bf location} person's location\\
{\bf name} person's name\\
{\bf serialVersionUID} serialization id\\
\end{verse}
}
\subsection{Constructor summary}{
\begin{verse}
{\bf Person()} \\
\end{verse}
}
\subsection{Method summary}{
\begin{verse}
{\bf displayInfo()} Display a person's information\\
{\bf kill()} Kill a person\\
\end{verse}
}
\subsection{Fields}{
\begin{itemize}
\item{
\index{serialVersionUID}
\label{personOfInterest.Person.serialVersionUID}private static final long {\bf  serialVersionUID}\begin{itemize}
\item{\vskip -.9ex 
serialization id}
\end{itemize}
}
\item{
\index{name}
\label{personOfInterest.Person.name} java.lang.String {\bf  name}\begin{itemize}
\item{\vskip -.9ex 
person's name}
\end{itemize}
}
\item{
\index{ifAlive}
\label{personOfInterest.Person.ifAlive} boolean {\bf  ifAlive}\begin{itemize}
\item{\vskip -.9ex 
person's living status}
\end{itemize}
}
\item{
\index{location}
\label{personOfInterest.Person.location} Location {\bf  location}\begin{itemize}
\item{\vskip -.9ex 
person's location}
\end{itemize}
}
\end{itemize}
}
\subsection{Constructors}{
\vskip -2em
\begin{itemize}
\item{ 
\index{Person()}
{\bf  Person}\\
\texttt{public\ {\bf  Person}()
\label{personOfInterest.Person()}}%end signature
}%end item
\end{itemize}
}
\subsection{Methods}{
\vskip -2em
\begin{itemize}
\item{ 
\index{displayInfo()}
{\bf  displayInfo}\\
\texttt{public void\ {\bf  displayInfo}()
\label{personOfInterest.Person.displayInfo()}}%end signature
\begin{itemize}
\item{
{\bf  Description}

Display a person's information
}
\end{itemize}
}%end item
\item{ 
\index{kill()}
{\bf  kill}\\
\texttt{public void\ {\bf  kill}()
\label{personOfInterest.Person.kill()}}%end signature
\begin{itemize}
\item{
{\bf  Description}

Kill a person
}
\end{itemize}
}%end item
\end{itemize}
}
}
\section{\label{personOfInterest.Player}\index{Player}Class Player}{
\vskip .1in 
Player who play the game in the virtual world\\ref: \url{http://docs.oracle.com/javase/tutorial/java/javaOO/index.html}\vskip .1in 
\subsection{Declaration}{
\small public class Player
\\ {\bf  extends} personOfInterest.Agent
\refdefined{personOfInterest.Agent}}
\subsection{Field summary}{
\begin{verse}
{\bf missionCompleted} number of completed missions\\
{\bf points} a player's points used in promotion system\\
{\bf serialVersionUID} serialization id\\
\end{verse}
}
\subsection{Constructor summary}{
\begin{verse}
{\bf Player(String, int, Location)} Construct a player with a name, a rank and a base\\
\end{verse}
}
\subsection{Method summary}{
\begin{verse}
{\bf demote()} Demote a player\\
{\bf promote()} Promote a player's rank according to his/her points\\
\end{verse}
}
\subsection{Fields}{
\begin{itemize}
\item{
\index{serialVersionUID}
\label{personOfInterest.Player.serialVersionUID}private static final long {\bf  serialVersionUID}\begin{itemize}
\item{\vskip -.9ex 
serialization id}
\end{itemize}
}
\item{
\index{missionCompleted}
\label{personOfInterest.Player.missionCompleted} int {\bf  missionCompleted}\begin{itemize}
\item{\vskip -.9ex 
number of completed missions}
\end{itemize}
}
\item{
\index{points}
\label{personOfInterest.Player.points} int {\bf  points}\begin{itemize}
\item{\vskip -.9ex 
a player's points used in promotion system}
\end{itemize}
}
\end{itemize}
}
\subsection{Constructors}{
\vskip -2em
\begin{itemize}
\item{ 
\index{Player(String, int, Location)}
{\bf  Player}\\
\texttt{public\ {\bf  Player}(\texttt{java.lang.String} {\bf  inName},
\texttt{int} {\bf  inRank},
\texttt{Location} {\bf  inBase})
\label{personOfInterest.Player(java.lang.String, int, personOfInterest.Location)}}%end signature
\begin{itemize}
\item{
{\bf  Description}

Construct a player with a name, a rank and a base
}
\item{
{\bf  Parameters}
  \begin{itemize}
   \item{
\texttt{inName} -- specified name}
   \item{
\texttt{inRank} -- specified rank}
   \item{
\texttt{inBase} -- specified base}
  \end{itemize}
}%end item
\end{itemize}
}%end item
\end{itemize}
}
\subsection{Methods}{
\vskip -2em
\begin{itemize}
\item{ 
\index{demote()}
{\bf  demote}\\
\texttt{public void\ {\bf  demote}()
\label{personOfInterest.Player.demote()}}%end signature
\begin{itemize}
\item{
{\bf  Description}

Demote a player
}
\end{itemize}
}%end item
\item{ 
\index{promote()}
{\bf  promote}\\
\texttt{public void\ {\bf  promote}()
\label{personOfInterest.Player.promote()}}%end signature
\begin{itemize}
\item{
{\bf  Description}

Promote a player's rank according to his/her points
}
\end{itemize}
}%end item
\end{itemize}
}
\subsection{Members inherited from class Agent }{
\texttt{personOfInterest.Agent} {\small 
\refdefined{personOfInterest.Agent}}
{\small 

\vskip -2em
\begin{itemize}
\item{\vskip -1.5ex 
\texttt{ {\bf  base}}%end signature
}%end item
\item{\vskip -1.5ex 
\texttt{ {\bf  calendar}}%end signature
}%end item
\item{\vskip -1.5ex 
\texttt{ {\bf  currentDay}}%end signature
}%end item
\item{\vskip -1.5ex 
\texttt{public void {\bf  displayInfo}()
}%end signature
}%end item
\item{\vskip -1.5ex 
\texttt{ {\bf  id}}%end signature
}%end item
\item{\vskip -1.5ex 
\texttt{public void {\bf  passDay}()
}%end signature
}%end item
\item{\vskip -1.5ex 
\texttt{public void {\bf  passDays}(\texttt{int} {\bf  days})
}%end signature
}%end item
\item{\vskip -1.5ex 
\texttt{ {\bf  rank}}%end signature
}%end item
\item{\vskip -1.5ex 
\texttt{public static final {\bf  RANKS}}%end signature
}%end item
\item{\vskip -1.5ex 
\texttt{public void {\bf  resetCal}()
}%end signature
}%end item
\item{\vskip -1.5ex 
\texttt{public void {\bf  resetCal}(\texttt{int} {\bf  days})
}%end signature
}%end item
\item{\vskip -1.5ex 
\texttt{private static final {\bf  serialVersionUID}}%end signature
}%end item
\end{itemize}
}
\subsection{Members inherited from class Person }{
\texttt{personOfInterest.Person} {\small 
\refdefined{personOfInterest.Person}}
{\small 

\vskip -2em
\begin{itemize}
\item{\vskip -1.5ex 
\texttt{public void {\bf  displayInfo}()
}%end signature
}%end item
\item{\vskip -1.5ex 
\texttt{ {\bf  ifAlive}}%end signature
}%end item
\item{\vskip -1.5ex 
\texttt{public void {\bf  kill}()
}%end signature
}%end item
\item{\vskip -1.5ex 
\texttt{ {\bf  location}}%end signature
}%end item
\item{\vskip -1.5ex 
\texttt{ {\bf  name}}%end signature
}%end item
\item{\vskip -1.5ex 
\texttt{private static final {\bf  serialVersionUID}}%end signature
}%end item
\end{itemize}
}
}
\section{\label{personOfInterest.POI}\index{POI}Class POI}{
\vskip .1in 
Persons of Interest\\ref: \url{http://docs.oracle.com/javase/tutorial/java/javaOO/index.html}\vskip .1in 
\subsection{Declaration}{
\small public class POI
\\ {\bf  extends} personOfInterest.Person
\refdefined{personOfInterest.Person}}
\subsection{Field summary}{
\begin{verse}
{\bf area} area of a POI\\
{\bf id} POI's id number\\
{\bf ifCriminal} true is the POI is criminal; false otherwise\\
{\bf serialVersionUID} serialization id\\
\end{verse}
}
\subsection{Constructor summary}{
\begin{verse}
{\bf POI(String, boolean)} Construct a POI with a name specifying if he/she is a criminal\\
\end{verse}
}
\subsection{Method summary}{
\begin{verse}
{\bf getID()} Get POI's ID\\
\end{verse}
}
\subsection{Fields}{
\begin{itemize}
\item{
\index{serialVersionUID}
\label{personOfInterest.POI.serialVersionUID}private static final long {\bf  serialVersionUID}\begin{itemize}
\item{\vskip -.9ex 
serialization id}
\end{itemize}
}
\item{
\index{ifCriminal}
\label{personOfInterest.POI.ifCriminal} boolean {\bf  ifCriminal}\begin{itemize}
\item{\vskip -.9ex 
true is the POI is criminal; false otherwise}
\end{itemize}
}
\item{
\index{id}
\label{personOfInterest.POI.id} int {\bf  id}\begin{itemize}
\item{\vskip -.9ex 
POI's id number}
\end{itemize}
}
\item{
\index{area}
\label{personOfInterest.POI.area} int {\bf  area}\begin{itemize}
\item{\vskip -.9ex 
area of a POI}
\end{itemize}
}
\end{itemize}
}
\subsection{Constructors}{
\vskip -2em
\begin{itemize}
\item{ 
\index{POI(String, boolean)}
{\bf  POI}\\
\texttt{public\ {\bf  POI}(\texttt{java.lang.String} {\bf  inName},
\texttt{boolean} {\bf  criminality})
\label{personOfInterest.POI(java.lang.String, boolean)}}%end signature
\begin{itemize}
\item{
{\bf  Description}

Construct a POI with a name specifying if he/she is a criminal
}
\item{
{\bf  Parameters}
  \begin{itemize}
   \item{
\texttt{inName} -- POI's name}
   \item{
\texttt{criminality} -- true if the POI is a criminal; false otherwise}
  \end{itemize}
}%end item
\end{itemize}
}%end item
\end{itemize}
}
\subsection{Methods}{
\vskip -2em
\begin{itemize}
\item{ 
\index{getID()}
{\bf  getID}\\
\texttt{public int\ {\bf  getID}()
\label{personOfInterest.POI.getID()}}%end signature
\begin{itemize}
\item{
{\bf  Description}

Get POI's ID
}
\end{itemize}
}%end item
\end{itemize}
}
\subsection{Members inherited from class Person }{
\texttt{personOfInterest.Person} {\small 
\refdefined{personOfInterest.Person}}
{\small 

\vskip -2em
\begin{itemize}
\item{\vskip -1.5ex 
\texttt{public void {\bf  displayInfo}()
}%end signature
}%end item
\item{\vskip -1.5ex 
\texttt{ {\bf  ifAlive}}%end signature
}%end item
\item{\vskip -1.5ex 
\texttt{public void {\bf  kill}()
}%end signature
}%end item
\item{\vskip -1.5ex 
\texttt{ {\bf  location}}%end signature
}%end item
\item{\vskip -1.5ex 
\texttt{ {\bf  name}}%end signature
}%end item
\item{\vskip -1.5ex 
\texttt{private static final {\bf  serialVersionUID}}%end signature
}%end item
\end{itemize}
}
}
\section{\label{personOfInterest.Problem}\index{Problem}Class Problem}{
\vskip .1in 
Problem that will occur in a mission\\ref: \url{http://docs.oracle.com/javase/tutorial/java/javaOO/index.html}\vskip .1in 
\subsection{Declaration}{
\small public class Problem
\\ {\bf  extends} java.lang.Object
\refdefined{java.lang.Object}\\ {\bf  implements} 
java.io.Serializable}
\subsection{Field summary}{
\begin{verse}
{\bf answer} a problem's answer; might be integer, string, boolean, etc.\\
{\bf chances} maximum chances to answer a problem\\
{\bf days} days that a mission used\\
{\bf description} details of a problem\\
{\bf goodMessage} message after good answer\\
{\bf location} location where a problem occurred\\
{\bf mission} mission that contains a problem\\
{\bf name} a problem's name\\
{\bf points} value of a problem in points (promotion system)\\
{\bf punishment} punishment\\
{\bf serialVersionUID} serialization id\\
{\bf type} type of a problem's answer; 0 for int, 1 for double, 2 for boolean, 3 for String\\
\end{verse}
}
\subsection{Constructor summary}{
\begin{verse}
{\bf Problem(String, String, int, Object, int, int)} Construct a problem\\
\end{verse}
}
\subsection{Method summary}{
\begin{verse}
{\bf getType()} Get a problem's type\\
{\bf ifCorrect(Object)} Check if a response from a source is correct\\
{\bf punishment()} Punishment for wrong answers\\
\end{verse}
}
\subsection{Fields}{
\begin{itemize}
\item{
\index{serialVersionUID}
\label{personOfInterest.Problem.serialVersionUID}private static final long {\bf  serialVersionUID}\begin{itemize}
\item{\vskip -.9ex 
serialization id}
\end{itemize}
}
\item{
\index{mission}
\label{personOfInterest.Problem.mission} Mission {\bf  mission}\begin{itemize}
\item{\vskip -.9ex 
mission that contains a problem}
\end{itemize}
}
\item{
\index{name}
\label{personOfInterest.Problem.name} java.lang.String {\bf  name}\begin{itemize}
\item{\vskip -.9ex 
a problem's name}
\end{itemize}
}
\item{
\index{description}
\label{personOfInterest.Problem.description} java.lang.String {\bf  description}\begin{itemize}
\item{\vskip -.9ex 
details of a problem}
\end{itemize}
}
\item{
\index{answer}
\label{personOfInterest.Problem.answer} java.lang.Object {\bf  answer}\begin{itemize}
\item{\vskip -.9ex 
a problem's answer; might be integer, string, boolean, etc.}
\end{itemize}
}
\item{
\index{type}
\label{personOfInterest.Problem.type} int {\bf  type}\begin{itemize}
\item{\vskip -.9ex 
type of a problem's answer; 0 for int, 1 for double, 2 for boolean, 3 for String}
\end{itemize}
}
\item{
\index{location}
\label{personOfInterest.Problem.location} Location {\bf  location}\begin{itemize}
\item{\vskip -.9ex 
location where a problem occurred}
\end{itemize}
}
\item{
\index{points}
\label{personOfInterest.Problem.points} int {\bf  points}\begin{itemize}
\item{\vskip -.9ex 
value of a problem in points (promotion system)}
\end{itemize}
}
\item{
\index{chances}
\label{personOfInterest.Problem.chances} int {\bf  chances}\begin{itemize}
\item{\vskip -.9ex 
maximum chances to answer a problem}
\end{itemize}
}
\item{
\index{goodMessage}
\label{personOfInterest.Problem.goodMessage} java.lang.String {\bf  goodMessage}\begin{itemize}
\item{\vskip -.9ex 
message after good answer}
\end{itemize}
}
\item{
\index{days}
\label{personOfInterest.Problem.days} int {\bf  days}\begin{itemize}
\item{\vskip -.9ex 
days that a mission used}
\end{itemize}
}
\item{
\index{punishment}
\label{personOfInterest.Problem.punishment} java.lang.String {\bf  punishment}\begin{itemize}
\item{\vskip -.9ex 
punishment}
\end{itemize}
}
\end{itemize}
}
\subsection{Constructors}{
\vskip -2em
\begin{itemize}
\item{ 
\index{Problem(String, String, int, Object, int, int)}
{\bf  Problem}\\
\texttt{public\ {\bf  Problem}(\texttt{java.lang.String} {\bf  inName},
\texttt{java.lang.String} {\bf  inDesc},
\texttt{int} {\bf  inType},
\texttt{java.lang.Object} {\bf  inAnswer},
\texttt{int} {\bf  inPoints},
\texttt{int} {\bf  inChances})
\label{personOfInterest.Problem(java.lang.String, java.lang.String, int, java.lang.Object, int, int)}}%end signature
\begin{itemize}
\item{
{\bf  Description}

Construct a problem
}
\item{
{\bf  Parameters}
  \begin{itemize}
   \item{
\texttt{inName} -- problem's name}
   \item{
\texttt{inDesc} -- problem's description}
   \item{
\texttt{type} -- type of problem answer}
   \item{
\texttt{answer} -- problem's answer}
  \end{itemize}
}%end item
\end{itemize}
}%end item
\end{itemize}
}
\subsection{Methods}{
\vskip -2em
\begin{itemize}
\item{ 
\index{getType()}
{\bf  getType}\\
\texttt{public java.lang.String\ {\bf  getType}()
\label{personOfInterest.Problem.getType()}}%end signature
\begin{itemize}
\item{
{\bf  Description}

Get a problem's type
}
\item{{\bf  Returns} -- 
the type of the problem 
}%end item
\end{itemize}
}%end item
\item{ 
\index{ifCorrect(Object)}
{\bf  ifCorrect}\\
\texttt{public boolean\ {\bf  ifCorrect}(\texttt{java.lang.Object} {\bf  response})
\label{personOfInterest.Problem.ifCorrect(java.lang.Object)}}%end signature
\begin{itemize}
\item{
{\bf  Description}

Check if a response from a source is correct
}
\item{
{\bf  Parameters}
  \begin{itemize}
   \item{
\texttt{response} -- the response from the user}
  \end{itemize}
}%end item
\item{{\bf  Returns} -- 
if the response is consistent with the answer 
}%end item
\end{itemize}
}%end item
\item{ 
\index{punishment()}
{\bf  punishment}\\
\texttt{public boolean\ {\bf  punishment}()
\label{personOfInterest.Problem.punishment()}}%end signature
\begin{itemize}
\item{
{\bf  Description}

Punishment for wrong answers
}
\end{itemize}
}%end item
\end{itemize}
}
}
\section{\label{personOfInterest.Saved}\index{Saved}Class Saved}{
\vskip .1in 
Saved file for the game\vskip .1in 
\subsection{Declaration}{
\small public class Saved
\\ {\bf  extends} java.lang.Object
\refdefined{java.lang.Object}\\ {\bf  implements} 
java.io.Serializable}
\subsection{Field summary}{
\begin{verse}
{\bf savedLocation} ocation saved\\
{\bf savedPlayer} player saved\\
{\bf serialVersionUID} serialization id\\
\end{verse}
}
\subsection{Constructor summary}{
\begin{verse}
{\bf Saved(Player)} Construct a Saved class containing player only\\
{\bf Saved(Player, Location)} Construct a Saved class containing player and location\\
\end{verse}
}
\subsection{Method summary}{
\begin{verse}
{\bf getLocation()} Get the saved location in a class\\
{\bf getPlayer()} Get the saved player in a class\\
\end{verse}
}
\subsection{Fields}{
\begin{itemize}
\item{
\index{serialVersionUID}
\label{personOfInterest.Saved.serialVersionUID}private static final long {\bf  serialVersionUID}\begin{itemize}
\item{\vskip -.9ex 
serialization id}
\end{itemize}
}
\item{
\index{savedPlayer}
\label{personOfInterest.Saved.savedPlayer}public static Player {\bf  savedPlayer}\begin{itemize}
\item{\vskip -.9ex 
player saved}
\end{itemize}
}
\item{
\index{savedLocation}
\label{personOfInterest.Saved.savedLocation}public static Location {\bf  savedLocation}\begin{itemize}
\item{\vskip -.9ex 
ocation saved}
\end{itemize}
}
\end{itemize}
}
\subsection{Constructors}{
\vskip -2em
\begin{itemize}
\item{ 
\index{Saved(Player)}
{\bf  Saved}\\
\texttt{public\ {\bf  Saved}(\texttt{Player} {\bf  player})
\label{personOfInterest.Saved(personOfInterest.Player)}}%end signature
\begin{itemize}
\item{
{\bf  Description}

Construct a Saved class containing player only
}
\item{
{\bf  Parameters}
  \begin{itemize}
   \item{
\texttt{player} -- the player intending to be saved}
  \end{itemize}
}%end item
\end{itemize}
}%end item
\item{ 
\index{Saved(Player, Location)}
{\bf  Saved}\\
\texttt{public\ {\bf  Saved}(\texttt{Player} {\bf  player},
\texttt{Location} {\bf  location})
\label{personOfInterest.Saved(personOfInterest.Player, personOfInterest.Location)}}%end signature
\begin{itemize}
\item{
{\bf  Description}

Construct a Saved class containing player and location
}
\item{
{\bf  Parameters}
  \begin{itemize}
   \item{
\texttt{player} -- the player intending to be saved}
   \item{
\texttt{location} -- the location intending to be saved}
  \end{itemize}
}%end item
\end{itemize}
}%end item
\end{itemize}
}
\subsection{Methods}{
\vskip -2em
\begin{itemize}
\item{ 
\index{getLocation()}
{\bf  getLocation}\\
\texttt{public Location\ {\bf  getLocation}()
\label{personOfInterest.Saved.getLocation()}}%end signature
\begin{itemize}
\item{
{\bf  Description}

Get the saved location in a class
}
\item{{\bf  Returns} -- 
the saved location 
}%end item
\end{itemize}
}%end item
\item{ 
\index{getPlayer()}
{\bf  getPlayer}\\
\texttt{public Player\ {\bf  getPlayer}()
\label{personOfInterest.Saved.getPlayer()}}%end signature
\begin{itemize}
\item{
{\bf  Description}

Get the saved player in a class
}
\item{{\bf  Returns} -- 
the saved player 
}%end item
\end{itemize}
}%end item
\end{itemize}
}
}
}
\printindex
}